\documentclass[main]{subfiles}
\begin{document}
\chapter{初步聚类的正确性}\label{chp:proof_greedy}
我们假设样本点满足\emph{半随机模型},即每个子空间是确定的,
但是采样点是在每个子空间中随机生成的。进一步地,
不妨设采样点都是单位向量,这相当于它们都均匀采样于自身子空间的单位球上。
简单起见,我们设每个子空间的唯独和点数相同,即$d_0\triangleq d_1=\cdots=d_L,
n_0\triangleq = n_1=\cdots=n_L$,且不考虑噪音。

令$D_i\in \R^{d\times d_0}, i\in [L]$为子空间$\S^{(i)}$的正交基组成的矩阵。两个
子空间的\emph{相似度}定义为
\begin{align*}
  \aff(i,j) \triangleq \frac{\|D_i^\top D_j\|_F}{\sqrt{d_0}} =
  \sqrt{\frac{\sum_{k=1}^{d} \cos^2 \theta^{i,j}_k}{d_0}} \in [0,1],
\end{align*}
其中 $\theta^{i,j}_k$ 是 $\D_i$ 和 $\D_j$之间的第$k$个主角。
子空间 $\D_i$ 和 $\D_j$ 完全一样当且仅当 $\aff(i,j) = 1$。
若 $\aff(i,j) = 0$,那么 $\D_i$ 中的任意向量垂直于 $\D_j$。
我们定义最大相似度为
$$ \max \aff \triangleq \max_{i,j \in [L], i \neq j} \aff(i,j) \in [0,1].$$

因此根据以上假设有 $n = n_0 L$ 个点,每个点 $y_i$ 独立同分布地均匀采样于
$\S^{d-1} \cap \set{D}_{l_i}$ , $\S^{d-1}$ 是 $\mathbb{R}^d$ 中的单位球。于是
\begin{align*}
  y_i = D_{l_i} x_i ,\quad x_i \sim \text{Unif}(\mathbb{S}^{d_0-1}) ,\quad
  \forall i \in [N].
\end{align*}

下面是 Levy 引理的特殊形式。
\begin{lemma}[\cite{ledoux2005concentration}] \label{lem:measureconc}
对 $x \sim \text{Unif}(\mathbb{S}^{d-1})$,我们有
\begin{align*} 
\Pr \{ \| A x \|_2  > m\|A x\|_2 + t \} &\le \exp \left( - \frac{dt^2}{2\|A\|_2^2} \right), \\
\Pr \{ \| A x \|_2  < m\|A x\|_2 - t \} &\le \exp \left( - \frac{dt^2}{2\|A\|_2^2} \right).
\end{align*}
对任意 $A \in \R^{p \times d}, t > 0$ 成立。 其中$m\|Ax\|_2$ 是 $\|Ax\|_2$ 的中位数。
\end{lemma}

\begin{lemma} \label{lem:expectation}
  对 $x \sim \text{Unif}(\mathbb{S}^{d-1})$ 和任意矩阵 
  $A \in \mathbb{R}^{p \times d}$,我们有
  $$ E[\|Ax\|_2^2] = \frac{1}{d} \|A\|_F^2 $$
\end{lemma}

\begin{proof}
令 $A = U \Sigma V^T$ 为 $A$ 的奇异值分解,于是
\begin{align*}
  E[\|Ax\|_2^2] = E[\|U \Sigma V^T x\|_2^2] 
  = E[\|\Sigma x\|_2^2] = \sum_{i=1}^{\min(p,d)} \sigma_i^2 E[x_i^2] 
  = \sum_{i=1}^{\min(p,d)} \sigma_i^2 \cdot \frac{1}{d} = \frac{1}{d} \|A\|_F^2.
\end{align*}
其中$ E[x_i^2] = \frac{1}{d}$,是因为 $x$ 的每个分量对称,期望相同,而$ \sum_{i=1}^d E[x_i^2] = 1 $,
\end{proof}

\begin{lemma} \label{lem:subspace_concentration}
  对 $x \sim \text{Unif}(\mathbb{S}^{d-1})$ 和任意矩阵 
  $A \in \mathbb{R}^{p \times d}$,若 $\|A\|_2$ 有界,我们有
  \begin{align*}
    \Pr \left\{ \|A x\|_2 > \sqrt{\frac{1}{d}} \|A\|_F + 
    \|A\|_2 \cdot \left( \sqrt{\frac{2\pi}{d}} + \delta \right) \right\}
    \le e^{-\frac{d \delta^2}{2}},
  \end{align*}
  其中$\delta$ 为任意正实数。
\end{lemma}

\begin{proof}
  由引理\ref{lem:measureconc},我们有
  \begin{align*}
    \left| E\|A x\|_2 - m\|A x\|_2 \right| &\le E\left[ \left| \|A x\|_2 - m\|A x\|_2 \right| \right] \le \int_0^\infty 2e^{-\frac{dt^2}{2\|A\|_2^2}} dt = \sqrt{\frac{2\pi}{d}} \|A\|_2.
  \end{align*}
  于是
  \begin{align*}
    \Pr \left\{ \|A x\|_2 > \sqrt{\frac{\|A\|_F^2}{d}} + \sqrt{\frac{2\pi}{d}} \|A\|_2 + t \right\}
    &= \Pr \left\{ \|A x\|_2 > \sqrt{E\|A x\|_2^2} + \sqrt{\frac{2\pi}{d}} \|A\|_2 + t \right\} \\
    &\le \Pr \left\{ \|A x\|_2 > E\|A x\|_2 + \sqrt{\frac{2\pi}{d}} \|A\|_2 + t \right\} \\
    &\le \Pr \left\{ \|A x\|_2 > m\|A x\|_2 + t \right\} \\
    &\le \exp \left( -\frac{dt^2}{2 \|A\|_2^2} \right).
  \end{align*}
  其中由Jensen 不等式 $ E\|A x\|_2 \le \sqrt{E\|Ax\|_2^2}$,
  由引理\ref{lem:subspace_concentration} $E \|Ax\|_2^2=\frac{\|A\|_F^2}{d}$。
  另$\delta = \frac{t}{\|A\|_2}$ 即得证。
\end{proof}

为了控制当前点与
\begin{lemma}[\cite{ball1997intro_convex_geometry}]\label{lem:spherical_cap}
  对 $x \sim \text{Unif}(\S^{d-1})$ 和任意固定的向量 $a$,我们有
  \begin{align*}
    \left( \frac{1-\varepsilon}{2} \right)^{\frac{d-1}{2}} \ge
    \Pr\left(|a^Tx|>\epsilon \|a\|_2\right) \leq 2e^{\frac{-d\epsilon^2}{2}}
  \end{align*}
\end{lemma}

\begin{lemma} \label{lem:projbound_upper}
  设 $x_1, \ldots, x_n $ 为采样于$\text{Unif}(\S^{d-1})$ 的独立同分布随机向量,
  令 $z$ 为 $\{ |a^T x_i|, 1 \le i \le n\}$
  中的最大值,其中$a \in \R^d $ 是固定向量,则
  \begin{align*}
    \Pr{z>\sqrt{\frac{2}{d} \log \frac{2n}{\delta}} \|a\|_2} \le \delta,
  \end{align*}
  其中$0<\delta<1$。
\end{lemma}

\begin{proof}
  对任意$\varepsilon>0$,我们有
  \begin{align*}
    \Pr{z > \varepsilon \|a\|_2} &\le \Pr\left\{\exists i \in [n] : |a^T
    x_i|>\varepsilon \|a\|_2 \right\} \\
    &\le n \Pr\left\{ |a^T x_i|>\varepsilon \|a\|_2\right\} \\
    &\le 2n e^{\frac{-d\varepsilon^2}{2}},
  \end{align*}
  其中第三个不等式用了引理\ref{lem:spherical_cap}。令$\delta=2n
  e^{\frac{-d\varepsilon^2}{2}}$即得证。
\end{proof}

\begin{lemma} \label{lem:projbound_lower}
  设 $x_1, \ldots, x_n $ 为采样于$\text{Unif}(\S^{d-1})$ 的独立同分布随机向量,
  令 $z_m$ 为 $\{ |a^T x_i|, 1 \le i \le n\}$ 中的第m大值,其中$a \in \R^d $ 是固定向量,则
  \begin{align*}
    \Pr\left\{z_m>\left( 1-2\left(\frac{m\log \frac{ne}{m}-\log \delta}{n-m+1}
    \right)^{\frac{2}{d-1}}\right) \|a\|_2\right\} \ge 1-\delta,
  \end{align*}
  其中$0<\delta<1$。
\end{lemma}
\begin{proof}
  对任意$\varepsilon>0$,我们有
  \begin{align*}
    \Pr\{z_m \le \varepsilon \|a\|_2\} &= \Pr\left\{\exists I \subset [n], |I|=n-m+1:
    z_i \le \varepsilon, \forall i \in I\right\} \\
    &\le \binom{n}{m-1} \cdot \Pr \left\{ z_1 \le \varepsilon \right\}^{n-m+1}\\
    &\le \left( \frac{ne}{m} \right)^m \cdot \left( 1-\left(
    \frac{1-\varepsilon}{2} \right)^{\frac{d-1}{2}} \right)^{n-m+1} \\
    &\le \exp \left\{ m \log \frac{ne}{m}-\left(
    \frac{1-\varepsilon}{2}\right)^{\frac{d-1}{2}} (n+m-1) \right\},
  \end{align*}
  其中我们用了引理\ref{lem:spherical_cap}, $\binom{n}{m}\le
  (\frac{ne}{m})^m$ 和 $1+x\le e^x, \forall x$。
  令
  $$\delta= \exp \left\{ m \log \frac{ne}{m}-\left(
    \frac{1-\varepsilon}{2}\right)^{\frac{d-1}{2}} (n+m-1)
  \right\}$$
  再取互补事件即可得证。
\end{proof}

为了保证算法\ref{alg:nsn}的正确性,我们要求对每个点,其$k$个邻居都与自身属于同一
子空间。不失一般性地,我们考虑$x_1$ ,上面的条件意味着
$$ |x_1^T x_i|>|x_1^T x_j| \forall j \in \{j:l_j \ne l_1\}, $$
其中 $x_i$ 为$l_1$类中在$x_1$上投影第$k$大的点。
\begin{theorem} \label{thm:nsn}
  对于上面的半随机模型,若存在$\delta_1, \delta_2, \delta_3 >0$ 使得
  $$\max \aff \le \frac{1-2\left( \frac{k\log \frac{(n-1)e}{k}-\log \delta_1}
  {n-k} \right)^{\frac{2}{d-1}}}{\left( 1+\sqrt{2\pi}+\sqrt{d_0}\delta_3\right)\delta_2}, $$
  则每个点与其$k$个投影最近邻都属于同一子空间的概率不小于$(1-\delta_1)(1-2\exp(\frac{-d_0
  \delta_2^2}{2}))(1-\exp(\frac{-d_0 \delta_3^2}{2}))$。
\end{theorem}
\begin{proof}
不失一般性地,我们考虑$x_1$ ,要满足的条件为
\begin{align}
  |y_1^T y_i|>|y_1^T y_j| \forall j \in \{j:l_j \ne l_1\},
  \label{eq:cond}
\end{align}
其中 $y_i$ 为$l_1$类中在$y_1$上投影第$k$大的点。由于
\begin{align*}
  |y_1^1 y_i|=
  \begin{cases}
    |x_1^T x_i| & l_i = l_1 \\
    |x_1^T D_1^T D_{l_i}x_i| & l_i \ne l_1
  \end{cases}
\end{align*}
所以(\ref{eq:cond})的左边就是$n-1$个$\S^{d_0-1}$上的均匀采样点在
$x_1$上投影的第$k$大值$z_k$,根据引理\ref{lem:projbound_lower},有不小于$1-\delta_1$
的概率
\begin{align}
  z_k>\left( 1-2\left(\frac{k\log \frac{(n-1)e}{k}-\log \delta_1}{n-k}
    \right)^{\frac{2}{d_0-1}}\right)
  \label{eq:left}
\end{align}
成立。而根据引理\ref{lem:spherical_cap}
\begin{align} \label{eq:right1}
  |x_1^T D_1^T D_{l_i}x_i| \le \delta_2 \|D_j^T D_1 x_1\|_2
\end{align}
成立的概率不小于$1-\exp(\frac{d_0 \delta_2^2}{2})$。
进一步地,根据引理\ref{lem:subspace_concentration},有不小于$1-\exp(\frac{-d\delta_3^2}{2})$
的概率有
\begin{align}
  \|D_j^T D_1 x_1\|_2 &\le \frac{1}{\sqrt{d_0}} \|D_j^TD_1\|_F +
  \|D_j^T D_1\|_2 \left( \sqrt{\frac{2\pi}{d_0}}+\delta_3 \right) \nonumber\\
  &\le \|D_j^TD_1\|_F \left(\frac{1+\sqrt{2\pi}}{\sqrt{d_0}}+\delta_3
  \right) \nonumber \\
  &\le \sqrt{d_0} \max \aff \left(\frac{1+\sqrt{2\pi}}{\sqrt{d_0}}+\delta_3
  \right).
  \label{eq:right2}
\end{align}
结合(\ref{eq:left}),(\ref{eq:right1})和(\ref{eq:right2}),即得证。

\end{proof}

\end{document}
