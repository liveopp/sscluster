\documentclass[main]{subfiles}
\begin{document}
\chapter{问题及算法}\label{chp:prob_setup}
本章我们具体地介绍受SSC启发的新算法。

虽然SSC是目前最优秀的子空间聚类算法之一,但是它有两个缺陷:第一,SSC将每个点用其他点表示,
这个过程点与点是互相独立的,因此没有考虑优化表达系数之间的一致性。如果我们将数据点看作
“字典”,那么理应有些“词”比较出众,同一类的点都应该尽量用相同的几个点表示。这样
很可能使之后的邻接图聚类更准确,因为同一类的点都与相同的几个点相连。第二,SSC
期望得到稀疏表示,然而对于之后的谱聚类算法,邻接矩阵反而不能太稀疏。如果每个点都与
较多的同类点相连,这样聚类的效果很可能更好。

基于上面的考虑,我们提出了基于 Multitasking 和Group
LASSO的稀疏子空间聚类。首先,我们考虑将数据点聚成许多小类,使得每个小类的点
都属于同一子空间。事实上,在很多实际应用中,空间距离相近的点很可能属于同一类,
比如在运动分割中,无论摄像头、物体如何运动,同一个刚体上相近的几个点,在每一
帧上的位置都会很近,因此在高维空间中依然相近。于是利用贪心的K近邻方法,可以
将数据点初步聚类。如果将同一小类的点组成矩阵,我们得到$X_1, X_2, \cdots X_m$。
对$i\in [m]$,$X_i$用其他数据点表示得到系数$C_i$,连同初步聚类结果,构造邻接矩阵,
最后再进行谱聚类。

另一方面,将每一类的下表集合记为$\Omega$,如果把SSC的$l_1$范数换成 group $l_1$
范数,要求每个点都用尽量少的小类表示,这样可以保证系数具有较多的类内点上非零,
进而构造邻接矩阵,进行谱聚类。

\section{初步聚类} 

第一步要将数据点分成若干组,使得每一组里的点属于同个子空间。为了做到这点
必须利用数据的空间性质,然而如果只考虑数据点之间的欧式距离,无疑忽略了线性空间的性质,
所以我们尝试用贪心的方法构造出每个点的局部子空间,进而考虑投影最近邻。

首先假设子空间都通过原点,那么我们可以将每个点归一化为单位向量。
如果只看一个点$x_0$,最有可能和$x_0$在同一子空间的点自然是其投影最近邻,
即在$x_0$方向投影最大的点,记作$x_1$。然后$x_0$,$x_1$和原点张成了一个
二维平面,可以找到新的投影最近邻$x_2$,进而将$x_2$加入寻找下一个最投影近邻。

下面给出计算每个点的拓展最近邻的算法
\begin{algorithm} \caption{拓展最近邻}
  \begin{algorithmic} \label{alg:nsn}
    \Require $n$ 个样本点 $\mathcal{X} = \{x_1,\ldots,x_n\}$, 所需邻居的个数
    $K$, 邻域子间维度 $k_{max}$.
    \Ensure 所有样本点邻居关系$W$
    \For {$i= 1,\ldots,n$} \Comment{遍历所有的点}
      \State $\set{S} \gets \{x_i\}$ \Comment{$x_i$点张成的子空间}
      \For {$k = 1,\ldots,K$} \Comment{依次将最近邻加入子空间集合}
        \State $x^* \gets \argmin_{x \in \cX \setminus \set{S}} d(x,\spa\{\set{S}\})$
        \If{$|\set{S}| < k_{max}$}
          \State $\set{S} \gets \set{S} \cup \{x^*\}$
        \EndIf
      \EndFor
      \State 根据$\set{S}$生成$W_i$
    \EndFor
  \end{algorithmic}
\end{algorithm}

得到每个点的K近邻之后,还要把点分成若干组,因为这些近邻之间有重叠,但是后面的Multtasking
和 Group LASSO方法要求每个点只能在一个组中。当然这里可以直接使用谱聚类,不过简单
起见,我们直接采用“先到先得“的原则,先找到一个未分组的点$x_1$,
将其K近邻中没有分组的分成一组,再找下一个未分组点,直到分完所有点。
实际计算中,为了消除取点带来的随机性,我们总是按照2-范数从大到小,依次选取,最终
得到每个分组的指标集的集合$\Omega$。

\section{Multitasking 稀疏方法}
原有的 SSC 方法,主要求解下面的优化问题
\begin{align}\label{eq:Lasso}
  \min_{c_i} \; &\|c_i\|_1+\frac{\lambda}{2}\|x_i-X_{-i}c_i\|^2.
\end{align}
在有了初步的分组信息后,我们转而求解新的优化问题
\begin{align} \label{eq:Multi}
  \min_{C_I}\; & \|C_I^T\|_{2, 1} + \frac{\lambda}{2} \|X_I - X_{I^c}C_I\|_F^2
\end{align}
其中$I\in \Omega$ 为某个分组的指标集。类似地,我们要用$C_I$组成邻接矩阵,进而
做谱分解。不过不同于SSC,在每个分组内部我们缺少邻接关系,因此取$\max C_I$为
组内两点的连接权重,构造出完整的邻接矩阵,再进行谱聚类。

\section{Group LASSO 方法}
定义范数$\|x\|_{\Omega}:=\sum_{I\in \Omega} \|x_I\|_2$,不同于\ref{eq:Lasso},Group LASSO 方法考虑下面的优化问题
\begin{align}
  \min_{c_i} \; & \|c_i\|_{\Omega} + \frac{\lambda}{2} \|x_i - X_{-i}
  c^{(i)}\|_2^2
  \label{eq:group}
\end{align}
其中$c_i$是$x_i$对应的系数向量。

\end{document}
