\documentclass{ctexart}
\usepackage{amsmath, amsthm, amssymb}

\begin{document}
考虑一个一般的凸优化问题:
\begin{equation}\label{eq:Opt_A_general}
  \begin{aligned}
    \quad \min_{C, E} \; &\|C^T\|_{2,1}+\frac{\lambda}{2}\|E\|^2_F \quad &s.t. \quad Y=Xc+E.
  \end{aligned}
\end{equation}
We state Lemma~\ref{lemma:OptimalCondition}, which extends Lemma~7.1 in \cite{soltanolkotabi2011geometric}.
\begin{lemma}\label{lemma:OptimalCondition}
  考虑矩阵$X\in \mathbb{R}^{d\times n}$和矩阵$Y \in \mathbb{R}^{d\times n}$.
  如果存在三元组 $(C,E,N)$ 满足 $Y=XC+E$, $C$ 的行支撑集 $S\subseteq T$,$N$满足
  \begin{align*}
    N^T X_i = \frac{C^T_i}{\| C^T_i\|_2} \quad \forall i \in S,  & N=\lambda E, \\
    \|N^T X_{T\cap S^{c}}\|_{2, \infty} \leq 1, & \|N^T X_{T^{c}}\|_{2, \infty}<1,
  \end{align*}

  则对\eqref{eq:Opt_A_general} 的任意最优解 $(C^{*},E^{*})$ 必然有
  $(C^{*})^T_i=\mathbf{0} \quad \forall i in T^c$.
\end{lemma}

\begin{proof}
  对最优解 $(C^{*},E^{*})$, 我们有:
  \begin{align}
    &\|(C^*)^T\|_{2,1}+\frac{\lambda}{2}\|E^*\|_F^2 \nonumber\\
    =& \|(C^*)^T_S\|_{2,1}+\|(C^*)^T_{T\cap
    S^c}\|_{2,1}+\|(C^*)^T_{T^c}\|_{2,1} + \frac{\lambda}{2} \|E^*\|^2_F\nonumber\\
    \geq&\|C^T_S\|_{2,1}+\langle norm(C_S),C^*_S-C_S\rangle+\|(C^*)^T_{T\cap
    S^c}\|_{2,1}+\|(C^*)^T_{T^c}\|_{2,1} + \frac{\lambda}{2} \|E\|^2_F +\langle
    \lambda E,E^*-E\rangle\nonumber\\
    =&\|C^T_S\|_{2,1}+\langle N,X_S(C^*_S-C_S)\rangle+\|(C^*)^T_{T\cap
    S^c}\|_{2,1}+\|(C^*)^T_{T^c}\|_{2,1} + \frac{\lambda}{2} \|E\|^2_F +\langle
    \lambda E,E^*-E\rangle\nonumber\\
    =&\|C^T_S\|_{2,1}+\frac{\lambda}{2} \|E\|_F^2+ \|(C^*)^T_{T\cap
    S^c}\|_{2,1}-\langle N,X_{T\cap S^c}C^*_{T\cap S^c}\rangle
    +\|(C^*)^T_{T^c}\|_{2,1}-\langle N,X_{T^c}C^*_{T^c}\rangle. \label{eq:lemma_tmp1}
  \end{align}
  其中 $\frac{\lambda}{2} \|E^*\|_F^2 \geq \frac{\lambda}{2} \|E\|_F^2 +\langle
  \lambda E,E^*-E\rangle$ 可由Cauchy–Schwarz不等式推得。
  最后一个等式成立是因为 $(C,E)$和$(C^*,E^*)$都是可行解, 因此$\langle\nu,A((C^*)^T-c)\rangle+\langle\nu,e^*-e\rangle = \langle\nu,A(C^*)^T+e^*-(Ac+e)\rangle=0$. Also, note that $\|c_S\|_1+\frac{\lambda}{2} \|e\|^2=\|c\|_1+\frac{\lambda}{2} \|e\|^2$.

  With the inequality constraints of $\nu$ given in the lemma statement, we have
  \begin{align*}
    \langle \nu,A_{T\cap S^c}((C^*)^T_{T\cap S^c})\rangle=&\langle A_{T\cap S^c}^T\nu,((C^*)^T_{T\cap S^c})\rangle
    \leq \|A^T_{T\cap S^{c}}\nu\|_{\infty}\|(C^*)^T_{T\cap S^c}\|_1\leq\|(C^*)^T_{T\cap S^c}\|_1.
  \end{align*}
  Substitute into \eqref{eq:lemma_tmp1}, we get:
  \begin{equation*}
    \|(C^*)^T\|_1+\frac{\lambda}{2} \|e^*\|^2 \geq \|c\|_1+\frac{\lambda}{2} \|e\|^2 +(1-\|A^T_{T^{c}}\nu\|_{\infty})\|(C^*)^T_{T^c}\|_1,
  \end{equation*}
  where $(1-\|A^T_{T^{c}}\nu\|_{\infty})$ is strictly greater than $0$.

  Using the fact that $((C^*)^T,e^*)$ is an optimal solution, $\|(C^*)^T\|_1+\frac{\lambda}{2} \|e^*\|^2\leq \|c\|_1+\frac{\lambda}{2} \|e\|^2$. Therefore, $\|(C^*)^T_{T^c}\|_1=0$ and $(c,e)$ is also an optimal solution. This concludes the proof.
\end{document}
