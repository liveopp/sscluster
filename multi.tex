\documentclass[main.tex]{subfiles}
\begin{document}
\chapter{确定情形下的证明}\label{chp:proof_multi}

%,\ref{thm:thm_random_noise},\ref{thm:semirandom}~and~\ref{thm:fullrandom}.
我们不直接分析\eqref{eq:Lasso},而是引入松弛变量$E_I$,将其转化为有限制的等价形式:
\begin{align}\label{eq:Opt_original}
  \mathbf{P}_0:\quad \min_{C_I, E_I} \;
  \|C_I^T\|_{2,1}+\frac{\lambda}{2}\|E_I\|_F^2 \quad
  s.t. \quad X^{(\ell)}_I=X_{I^c}C_I+E_I.
\end{align}
The constraint can be rewritten as
\begin{equation}\label{eq:Opt_original_equi}
  y^{(\ell)}_i+z^{(\ell)}_i=(Y_{I^c}+Z_{I^c})c_i+e_i.
\end{equation}
The dual program of \eqref{eq:Opt_original} is:
\begin{equation}\label{eq:Opt_original_dual}
  \begin{aligned}
    \mathbf{D}_0:\quad \max_{N} \; \langle x_i,N \rangle - \frac{1}{2\lambda}N^TN \quad
    s.t. \quad \|(X_{I^c})^TN\|_{\infty} \leq 1.
  \end{aligned}
\end{equation}
Recall that we want to establish the conditions on noise magnitude $\delta$, structure of the data ($\mu$ and $r$ in the deterministic model and affinity in the semi-random model), and ranges of valid $\lambda$ such that by Definition~\ref{def:lasso_detection}, the solution $c_i$ is \emph{non-trivial} and has support indices inside the column set $X^{(\ell)}_{I^c}$ (i.e., satisfies SEP).
考虑一个一般的凸优化问题:
\begin{align}\label{eq:Opt_A_general}
  \quad \min_{C, E} \; &\|C^T\|_{2,1}+\frac{\lambda}{2}\|E\|^2_F \quad &s.t. \quad Y=XC+E.
\end{align}
We state Lemma~\ref{lemma:OptimalCondition}, which extends Lemma~7.1 in \cite{soltanolkotabi2011geometric}.
\begin{lemma}\label{lemma:OptimalCondition}
  考虑矩阵$X\in \mathbb{R}^{d\times n}$和矩阵$Y \in \mathbb{R}^{d\times n}$.
  如果存在三元组 $(C,E,N)$ 满足 $Y=XC+E$, $C$ 的行支撑集 $S\subseteq T$,$N$满足
  \begin{equation*}
    \begin{array}{cc}
      N^T X_S = \norm(C_S^T),  & N=\lambda E, \\
      \|N^T X_{T\cap S^{c}}\|_{2, \infty} \leq 1, & \|N^T X_{T^{c}}\|_{2, \infty}<1,
    \end{array}
  \end{equation*}
  则\eqref{eq:Opt_A_general} 的任意最优解 $(C^{*},E^{*})$ 必然有
  $(C^{*})^T_i=\mathbf{0} \quad \forall i \in T^c$.
\end{lemma}

\begin{proof}
  对最优解 $(C^{*},E^{*})$, 我们有:
  \begin{align}
    &\|(C^*)^T\|_{2,1}+\frac{\lambda}{2}\|E^*\|_F^2 \nonumber\\
    =& \|(C^*)^T_S\|_{2,1}+\|(C^*)^T_{T\cap
    S^c}\|_{2,1}+\|(C^*)^T_{T^c}\|_{2,1} + \frac{\lambda}{2} \|E^*\|^2_F\nonumber\\
    \geq&\|C^T_S\|_{2,1}+\langle \norm(C_S),C^*_S-C_S\rangle+\|(C^*)^T_{T\cap
    S^c}\|_{2,1}+\|(C^*)^T_{T^c}\|_{2,1} + \frac{\lambda}{2} \|E\|^2_F +\langle
    \lambda E,E^*-E\rangle\nonumber\\
    =&\|C^T_S\|_{2,1}+\langle N,X_S(C^*_S-C_S)\rangle+\|(C^*)^T_{T\cap
    S^c}\|_{2,1}+\|(C^*)^T_{T^c}\|_{2,1} + \frac{\lambda}{2} \|E\|^2_F +\langle
    \lambda E,E^*-E\rangle\nonumber\\
    =&\|C^T_S\|_{2,1}+\frac{\lambda}{2} \|E\|_F^2+ \|(C^*)^T_{T\cap
    S^c}\|_{2,1}-\langle N,X_{T\cap S^c}C^*_{T\cap S^c}\rangle
    +\|(C^*)^T_{T^c}\|_{2,1}-\langle N,X_{T^c}C^*_{T^c}\rangle. \label{eq:lemma_tmp1}
  \end{align}
  其中 $\frac{\lambda}{2} \|E^*\|_F^2 \geq \frac{\lambda}{2} \|E\|_F^2 +\langle
  \lambda E,E^*-E\rangle$ 可由Cauchy–Schwarz不等式推得。
  最后一个等式成立是因为 $(C,E)$和$(C^*,E^*)$都是可行解, 因此$\langle
  N,X(C^*-C)\rangle+\langle N,E^*-E\rangle = \langle
  N,XC^*+E^*-(XC+E)\rangle=0$. 同时注意到 $\| C^T_S\|_{2, 1} = \|C^T\|_{2, 1}$.

  使用 $N$ 的不等式条件, 我们有
  \begin{align*}
    \langle N,A_{T\cap S^c}((C^*)^T_{T\cap S^c})\rangle=&\langle A_{T\cap S^c}^TN,((C^*)^T_{T\cap S^c})\rangle
    \leq \|A^T_{T\cap S^{c}}N\|_{\infty}\|(C^*)^T_{T\cap S^c}\|_1\leq\|(C^*)^T_{T\cap S^c}\|_1.
  \end{align*}
  将其带入 \eqref{eq:lemma_tmp1}, 我们得到:
  \begin{equation*}
    \|(C^*)^T\|_{2,1}+\frac{\lambda}{2} \|E^*\|^2_F \geq \|C^T\|_{2,1}+ \frac{\lambda}{2} 
    \|E\|_F^2 +(1-\|N^T X_{T^{c}}\|_{2, \infty})\|(C^*)^T_{T^c}\|_{2,1},
  \end{equation*}
  其中 $(1-\|N^T X_{T^{c}}\|_{2, \infty})$ 严格大于 $0$.

  由于 $(C^*,e^*)$ 是最优解, $\|(C^*)^T\|_{2,1}+\frac{\lambda}{2}
  \|E^*\|_F^2\leq \|C^T\|_{2, 1}+\frac{\lambda}{2} \|E\|_F^2$. 
  因此 $\|(C^*)^T_{T^c}\|_{2,1}=0$ 且 $(C,E)$ 也是最优解.
\end{proof}

下一步是取$Y=X_I^{(\ell)}, X=X_{I^c}$,构造三元组 $(C,E,N)$ ,
其中 $N$ 满足引理\ref{lemma:OptimalCondition}中所需条件,而 $C$ 满足 SEP. 
这样所有 \eqref{eq:Opt_original} 的最优解都会满足 SEP.

\subsection{构造三元组}\label{sec:construct_nu}
为了构造满足条件的三元组,我们考虑下面 {\em
假想}的优化问题。\footnote{由于我们不知道每个点的类别,所以实际上无法求解这个问题,因此叫做``假想''}.

\begin{align}\label{eq:Opt_fictitious2}
  \mathbf{P}_1: \quad \min_{C^{(\ell)}_I, E_I} \;
  &\|(C^{(\ell)}_I)^T\|_{2,1}+\frac{\lambda}{2}\|E_I\|_F^2 \quad
  s.t. \quad Y^{(\ell)}_I+Z_I=(Y^{(\ell)}_{I^c}+Z^{(\ell)}_{I^c})C^{(\ell)}_I+E_I;
\end{align}
\begin{align}\label{eq:dual_fictitious2}
  \mathbf{D}_1: \quad \max_{N} \; &\langle X_I^{(\ell)},N \rangle -
  \frac{1}{2\lambda} \|N\|_F^2 \quad
  s.t. \quad \|N^T X^{(\ell)}_{I^c}\|_{2, \infty} \leq 1.
\end{align}

\eqref{eq:Opt_fictitious2}是有可行解的(我们只需要任取$C_I$,算出相应的$E_I$即可。 因此由强对偶性可得, 
对 \eqref{eq:Opt_fictitious2} 的最优解 $(C, E)$,存在 $N$ 满足:
\begin{align*}
  &\|N^T ((Y^{(\ell)}_{I^c})_{S^{c}}^T +(Z^{(\ell)}_{I^c})_{S^{c}}^T)\|_{2, \infty}\leq 1, \quad N=\lambda E, \\
  &\quad N^T ((Y^{(\ell)}_{I^c})_{S} +(Z^{(\ell)}_{I^c})_{S}) =\norm(C_S^T).
\end{align*}
只需要将$C$ 按行扩充一些零向量就可以得到 \eqref{eq:Opt_original} 的可行解,即
\begin{equation}\label{eq:candidate_sol}
\begin{cases}
  C_I= [\mathbf{0},...,\mathbf{0},(C_I^{(\ell)})^T,\mathbf{0},...,\mathbf{0}]^T
  \text{其中} C_I^{(\ell)}=C,\\
    E_I= E.
\end{cases}
\end{equation}
这样三元组$(C_I, E_I, N)$ 满足了引理~\ref{lemma:OptimalCondition} 的条件,除了
\begin{equation*}
    \left\|N^T [X_1,...,X_{\ell-1},X_{\ell+1},...,X_L]\right\|_{2,\infty}<1,
\end{equation*}
即我们要验证 $\forall x \in \mathcal{X}\setminus \mathcal{X}^{\ell}$,有
\begin{equation}\label{eq:dual_separation_condition}
    \| N^T x \|_2 < 1.
\end{equation}
因此,如果我们能证明 \eqref{eq:dual_fictitious2} 的解$N$ 满足 \eqref{eq:dual_separation_condition},
那么根据引理~\ref{lemma:OptimalCondition} 我们得到 \eqref{eq:candidate_sol}
这个可行解就是 \eqref{eq:Opt_original} 的最优解,于是 SEP 成立。

\subsection{对偶矩阵的限制}\label{sec:dual_separation}
%Recall that in order for SEP to hold, we need the constructed dual certificate $N$ in Section~\ref{sec:construct_nu} to obey \eqref{eq:dual_separation_condition} for all data point $x \in \mathcal{X}\setminus \mathcal{X}^{\ell}$.

Our strategy to show \eqref{eq:dual_separation_condition} is to provide an upper bound of $|\langle x, N \rangle|$ then impose the inequality on the upper bound.

我们将 $N$ 的每一列在子空间 $\mathcal{S}_{\ell}$ 上投影,得到 $N_{\parallel} :=\mathbb{P}_{S_\ell}(N)$,
$N_{\perp} := \mathbb{P}_{\mathcal{S}_{\ell}^\perp}(N)$。因此
\begin{align}\label{eq:showing_dual_sep_cond}
  \| N^T x \|_2 =& \| N^T (y+z)\|_2 \leq \| N_{\parallel}^T y\|_2+\|
  N_{\perp}^T y\|_2+\|N^T z\|_2\\
  \leq& \mu(\mathcal{X}_{\ell}) \|N_{\parallel}\|_2 + \|N_{\perp}\|_2\|y\|_2
  + \|z\|\|N\||\cos(\angle (z,N))|.
\end{align}
To see the last inequality, check that by Definition~\ref{def:incoherence}, $|\langle y,\frac{N_1}{\|N_1\|}\rangle| \leq\mu(\mathcal{X}_{\ell})$.

Since we are considering general (possibly adversarial) noise, we will use the relaxation $|\cos(\theta)|\leq 1$ for all cosine terms (a better bound under random noise will be given later). Thus, what left is to bound $\|N_1\|$ and $\|N_{\parallel}\|$ (note $\|N\|=\sqrt{\|N_1\|^2+\|N_{\parallel}\|^2} \leq \|N_1\|+\|N_{\parallel}\|$).

\subsubsection{控制 $\|N_{\parallel}\|$}
我们考察$N$ 在 \eqref{eq:dual_fictitious2}中的可行区域:
$$\left\{N \middle| \|(N^T X^{(\ell)}_{I^c})\|_{2,\infty} \leq 1\right\},$$
等价于
$$\left\{N \middle| \| N^T x_j\|_2 \leq 1 \quad\text{对$X^{(\ell)}_{I^c}$的每一列 $x_j$ }\right\}.$$
分解可得
$$\| N_{\parallel}^T y_j+N_{\parallel}^T (\mathbb{P}_{\mathcal{S}_{\ell}}z_j +
N_{\perp}^T z_j\|_2 \leq 1.$$
由三角不等式
\begin{equation}\label{eq:relax_constraint}
  \| N_{\parallel}^T y_j+N_{\parallel}^T (\mathbb{P}_{\mathcal{S}_{\ell}}z_j \|_2
  \leq 1+\|N_{\perp}^T z_j\|_2 \leq 1+\delta\|N_{\perp}\|_2.
\end{equation}
The relaxed condition contains the feasible region of $N_1$ in \eqref{eq:dual_fictitious2}.
%the geometric properties of symmetric convex polytope
It turns out that the geometric properties of this relaxed feasible region provides an upper bound of $\|N_1\|$.
根据多胞体的几何性质,我们有
\begin{align}
  & \| N_{\parallel}^T (y_j+ (\mathbb{P}_{\mathcal{S}_{\ell}}z_j) \|_2 \leq
  1+\delta\|N_{\perp}\|_2 \nonumber\\
  \Leftrightarrow & \|N_{\parallel}^T x\| \leq 1 \quad \forall x \in \mathcal{P}\left(\frac{Y_{I^c}^{(\ell)}+
  \mathbb{P}_{\mathcal{S}_{\ell}}(Z_{I^c}^{(\ell)})}{1+\delta\|N_{\perp}\|_2}\right)
  \nonumber\\
  \Rightarrow & \|N_{\parallel}^T x\| \leq 1 \quad \forall x \in \mathcal{P}\left(\frac{Y_{I^c}^{(\ell)}+
  \mathbb{P}_{\mathcal{S}_{\ell}}(Z_{I^c}^{(\ell)})}{1+\delta\|N_{\perp}\|_2}\right)
  \text{的最大内接球}.\label{eq:Geometric_dual}
\end{align}
注意到$\cP(\dot)$ 是对称凸包,则 $\cP(\dot)$ 的最大内接球中心在原点。
我们有如下引理。

\begin{lemma}\label{lemma:circum_inradius}
  若对于矩阵 $A \in \R^{m \times n}$, 存在 $r>0$,使得
  $$\|Ax\|_2 \leq 1 \quad \forall x \in \cB(0, r),$$
  其中$\cB(0, r)$ 表示 $\R^n$ 中的半径为$r$的欧式球。
  那么$\|A\|_2 \leq \frac{1}{r}$。
\end{lemma}
\begin{proof}
  设$A$ 的svd为$U\Sigma V^T$,取$x=rV_1$,则
  $$\|Ax\|_2 = r \|U \Sigma V^T V_1\|_2 = r \sigma_1 \leq 1.$$
  所以$\|A\|_2 = \sigma_1 \leq \frac{1}{r}$。
\end{proof}

\begin{lemma}\label{lemma:Y_containing_set}
  若矩阵 $X=Y+Z$,令 $\rho:=\max_{i}\|\mathbb{P}_\mathcal{S}z_i\|$, 且 $Y_i \in \mathcal{S}$ 其中
  $\mathcal{S}$ 是线性子空间, 那么我们有:
  \begin{equation*}
    r(\mathrm{Proj}_\mathcal{S} (\mathcal{P}(X))) \geq r(\mathcal{P}(Y)) - \rho
  \end{equation*}
\end{lemma}
\begin{proof}
  首先注意到投影算子是线性算子, 因此 $\mathrm{Proj}_\mathcal{S}(\mathcal{P}(X))=\mathcal{P}(\mathbb{P}_\mathcal{S} X)$。
  根据定义 $\mathcal{P}(\mathbb{P}_\mathcal{S} X)$ 的边界是集合 $\mathcal{B}:=
  \left\{y\text{ }|\text{ }y=\mathbb{P}_\mathcal{S} X c; \|c\|_1=1\right\}$。
  内接球是凸包内的最大球,因此 $r(\mathcal{P}(\mathbb{P}_\mathcal{S} X)) =
  \min_{y\in \mathcal{B}} \|y\|$。 对$y \in \mathcal{B} $ 我们给出下界:
  \begin{align*}
    \|y\| \geq& \|Yc\|-\|\mathbb{P}_\mathcal{S}Z c\|\geq r(\mathcal{P}(Y)) - {\sum}_j{\|\mathbb{P}_\mathcal{S}z_j}\||c_j|
    \geq r(\mathcal{P}(Y)) - \rho\|c\|_1.
  \end{align*}
  因此得证。
\end{proof}

根据\eqref{eq:Geometric_dual},引理~\ref{lemma:circum_inradius} 和引理~\ref{lemma:Y_containing_set},
我们可给出$N_{\parallel}$的上界:
\begin{align}
  \|N_{\parallel}\|_2 \leq& \frac{1+\delta\|N_{\perp}\|}
  {r(\mathcal{P}(Y_{I^c}^{(\ell)}+\mathbb{P}_{\mathcal{S}_{\ell}}(Z_{I^c}^{(\ell)}))}
  \nonumber \\
  =& \frac{1+\delta\|N_{\perp}\|}{r(\mathrm{Proj}_\mathcal{S_{\ell}}
  (\mathcal{P}(X_{I^c}^{(\ell)})))} \nonumber \\
  \leq& \frac{1+\delta\|N_{\perp}\|}{r{\left( \mathcal{Q}_{I^c}^{\ell}\right)}-\delta_1}.\label{eq:nu1_bound}
\end{align}
这个上界依赖于 $\|N_{\perp}\|_2$, 接下来我们将对其进行分析。


\subsubsection{控制$\|N_{\perp}\|_2$}

由于 $N$ 是 $\mathbf{D}_1$ 的最优解,由对偶问题的性质,我们有
$$ N=\lambda E=\lambda(X_I-X^{(\ell)}_{I^c}C). $$
将 $N$ 投影到 $\mathcal{S}^{\perp}_{\ell}$, 我们得到
$N_{\perp}=\lambda \mathbb{P}_{\mathcal{S}_{\ell}^{\perp}}(X_I-X^{(\ell)}_{I^c}C)
= \lambda \mathbb{P}_{\mathcal{S}_{\ell}^{\perp}}(Z_I-Z^{(\ell)}_{I^c}C)^2$.
于是
\begin{align}
  \|N_{\perp}\|_2 & = \max_{\|y\|_2=1} \|N_{\perp}^T y\|_2 \nonumber \\
  & \leq\lambda \left(\|\mathbb{P}_{\mathcal{S}_{\ell}^{\perp}}Z_I\|_2
  +\|C^T (\mathbb{P}_{\mathcal{S}_{\ell}^{\perp}}Z^{(\ell)}_{I^c})^T y\|_2\right)\nonumber\\
  & \leq \lambda\left( \sqrt{m} \delta + \|C^T \ty\|_2\right) \nonumber\\
  & \leq \lambda\left( \sqrt{m} \delta + \sum_j |\ty_j|\|C^T_j\|_2\right) \nonumber\\
  & \leq \lambda\left( \sqrt{m} \delta + \|\ty\|_{\infty}\|C^T\|_{2,1}\right) \nonumber\\
  & \leq \lambda \delta(\Cgl+\sqrt{m}) \leq \lambda\delta\sqrt{m}(\Cgl+1). \label{eq:bounding_nu2}
\end{align}
其中$m$ 为指标集 $I$中的元素个数,$\ty=\mathbb{P}_{\mathcal{S}_{\ell}^{\perp}}Z^{(\ell)}_{I^c})^Ty$,
因此$\|\ty\|_{\infty} \leq \|z_i\|_2 \|y\|_2 \leq \delta$。

下面我们给出 $\Cgl$的上界。
因为 $(C,E)$ 是 \eqref{eq:Opt_fictitious2} 的最优解,所以对任何可行解 $(\tC ,\tE)$
都有 $\Cgl+\frac{\lambda}{2}\|E\|^2_F \leq \tCgl+\frac{\lambda}{2}\|\tilde{E}\|_F^2$。
令 $\tC$ 为下面优化问题的解
\begin{align}\label{eq:Opt_y only}
\min_{c} \; \Cgl \quad
s.t. \quad Y^{(\ell)}_I=Y^{(\ell)}_{I^c}C,
\end{align}
由强对偶性,
$$\tCgl = \max_{N}\left\{\langle N,Y^{(\ell)}_I\rangle | \|N^T
Y^{(\ell)}_{I^c}\|_{2, \infty}\leq 1\right\}.$$
根据引理~\ref{lemma:circum_inradius}, 对偶问题的最优解 $\tilde{N}$ 满足
$\|\tilde{N}\|_2 \leq \frac{1}{r(\mathcal{Q}_{I^c}^{\ell})}$。于是
$$\tCgl = \langle\tilde{N},Y^{(\ell)}_I\rangle \leq \frac{m}{r(\mathcal{Q}_{I^c}^{\ell})}.$$

同时 $\tilde{E} = Z_I - Z_{I^c}^{(\ell)}\tC$, 所以 
$$\|\tilde{E}\|_F^2 \leq (\|Z_I\|+\sum_{j,k} \|z_j\||\tC_{j,k}|)^2
\leq \delta^2 (\sqrt{m} + \|\tC\|_1)^2\leq m \delta^2(1+ \tCgl),$$
于是
$$\Cgl \leq \tCgl + \frac{\lambda}{2}\|\tilde{E}\|_F^2-\frac{\lambda}{2}\|E\|_F^2
\leq \frac{m}{r{(\mathcal{Q}_{I^c}^{\ell})}}+\frac{\lambda}{2}m\delta^2
\left[1+\frac{m}{r{(\mathcal{Q}_{I^c}^{\ell})}}\right]^2-\frac{1}{2\lambda}\|N_{\perp}\|_2^2.$$
注意到 $\frac{\lambda}{2}\|E\|_F^2=\frac{1}{2\lambda}\|N\|_F^2
\geq\frac{1}{2\lambda}\|N_{\perp}\|_F^2 \geq\frac{1}{2\lambda}\|N_{\perp}\|_2^2$。
将 $\tCgl$ 的界带入 \eqref{eq:bounding_nu2} 
\begin{align*}
  &&\;&\|N_{\perp}\|_2 \leq \lambda \delta \sqrt{m} 
  \left(\frac{m}{r(\mathcal{Q}_{I^c}^{\ell})}+\frac{\lambda}{2}\delta^2m
  \left[1+\frac{m}{r(\mathcal{Q}_{I^c}^{\ell})}\right]^2+1\right)
  -\frac{\delta}{2}\sqrt{m}\|N_{\perp}\|_2^2\\
  \Leftrightarrow&&\;
  &\|N_{\perp}\|_2+\frac{\delta}{2}\sqrt{m}\|N_{\perp}\|_2^2\leq
  \lambda\delta\sqrt{m}\left(\frac{m}{r(\mathcal{Q}_{I^c}^{\ell})}+1\right)+
  \frac{\delta}{2}\sqrt{m} \left[\lambda\delta\sqrt{m}
  \left(\frac{m}{r(\mathcal{Q}_{I^c}^{\ell})}+1\right)\right]^2 .
\end{align*}
由于函数 $f(\alpha)=\alpha+\frac{\delta}{2}\sqrt{m}\alpha^2$ 在 $\alpha>0$
时单调递增,则上面的不等式可以推出
\begin{equation}\label{eq:nu2_bound}
  \|N_{\perp}\|_2 \leq \lambda\delta\sqrt{m}\left(\frac{m}{r(\mathcal{Q}_{I^c}^{\ell})}+1\right),
\end{equation}
即我们需要的 $\|N_{\perp}\|_2$ 的上界。

\subsubsection{ $\|N^T x\|_2<1$ 所需的条件}
%\vspace{15pt}
%\noindent{\large \textbf{Conditions for $|\langle x, N \rangle|<1$:} }
结合 \eqref{eq:showing_dual_sep_cond}, \eqref{eq:nu1_bound} 和 \eqref{eq:nu2_bound}, 
我们可以得出 $\|N^T x\|_2$ 的上界:
\begin{align*}
  &\|N^T x\|_2 \leq (\mu(\mathcal{X}_{\ell})+\|\mathbb{P}_{\mathcal{S}_{\ell}}z\|) 
  \|N_{\parallel}\|_2+(\|y\|+\|\mathbb{P}_{\mathcal{S}_{\ell}^{\perp}}z\|)\|N_{\perp}\|_2\\
  \leq&\frac{\mu(\mathcal{X}_{\ell})+\delta_1}{r{\left( \mathcal{Q}_{I^c}^{\ell}\right)}-\delta_1}
  +\left(\frac{(\mu(\mathcal{X}_{\ell})+\delta_1)\delta}{r{\left( \mathcal{Q}_{I^c}^{\ell}\right)}-\delta_1}+1+\delta\right)
  \|N_{\perp}\|_2\\
  \leq& \frac{\mu(\mathcal{X}_{\ell})+\delta_1}{r{\left( \mathcal{Q}_{I^c}^{\ell}\right)}-\delta_1} +
  \lambda\delta(1+\delta)\sqrt{m} \left(\frac{m}{r(\mathcal{Q}_{I^c}^{\ell})}+1\right)
  + \frac{\lambda\delta^2\sqrt{m}(\mu(\mathcal{X}_{\ell})+\delta_1)}
  {r{\left( \mathcal{Q}_{I^c}^{\ell}\right)}-\delta_1}\left(\frac{m}{r(\mathcal{Q}_{I^c}^{\ell})}+1\right).
\end{align*}
简单起见,我们把第二个 $r(\mathcal{Q}_{I^c}^{\ell})$ 松弛成 $r(\mathcal{Q}_{I^c}^{\ell})-\delta_1$。
这样对偶分离条件就被下式保证
\begin{align*}
  \frac{\mu(\mathcal{X}_{\ell})+\delta_1 +\lambda m^{3/2}\delta(1+\delta)+
  \lambda\sqrt{m}\delta^2(\mu(\mathcal{X}_{\ell})+\delta_1)}
  {r{\left( \mathcal{Q}_{I^c}^{\ell}\right)}-\delta_1}
  + \lambda\sqrt{m}\delta(1+\delta)+\frac{\lambda\delta^2 m^{3/2}(\mu(\mathcal{X}_{\ell})+\delta_1)}
  {r{\left( \mathcal{Q}_{I^c}^{\ell}\right)}(r{\left( \mathcal{Q}_{I^c}^{\ell}\right)}-\delta_1)}  < 1.
\end{align*}
记 $\rho:=\lambda\sqrt{m}\delta$, 假设 $\delta<r{\left( \mathcal{Q}_{I^c}^{\ell}\right)}$,
$(\mu(\mathcal{X}_{\ell})+\delta_1)<1$ ,那么我们有下面的不等式
\begin{align*}
  \frac{\lambda\sqrt{m}\delta^2(\mu(\mathcal{X}_{\ell})+\delta_1)}
  {r{\left( \mathcal{Q}_{I^c}^{\ell}\right)}-\delta_1}
  +\frac{\lambda\delta^2 m^{3/2}(\mu(\mathcal{X}_{\ell})+\delta_1)}
  {r{\left( \mathcal{Q}_{I^c}^{\ell}\right)}(r{\left( \mathcal{Q}_{I^c}^{\ell}\right)}-\delta_1)}
  < \frac{\rho (m+\delta)}{r{\left( \mathcal{Q}_{I^c}^{\ell}\right)}-\delta_1}.
\end{align*}
将上式带入化简,我们就得到了一个充分条件
%\begin{equation}\label{eq:dual_cond_simp}
%    \mu < \left(1-\rho\right)r(\mathcal{Q}_{I^c}^{\ell}) +\delta_1\rho-\rho-2\delta_1.
%\end{equation}
\begin{equation}\label{eq:dual_cond_simp}
  \mu(\mathcal{X}_{\ell}) +\rho [2m+(1+m)\delta] +\delta_1
  < \left(1-\rho(1+\delta)\right)(r(\mathcal{Q}_{I^c}^{\ell})-\delta_1).
\end{equation}
把 \eqref{eq:dual_cond_simp} 一般化到所有子空间的所有数据点上,
则必须对每个$\ell = 1,...,k$ 成立:
\begin{equation}\label{eq:Thm1_all}
  \mu(\mathcal{X}_{\ell}) +\rho [2m+(1+m)\delta] +\delta_1
  < \left(1-\rho(1+\delta)\right)\left(\min_{\{I: I\in I(\ell)\}}r(\mathcal{Q}^{(\ell)}_{I^c})-\delta_1\right).
\end{equation}
This gives a first condition on $\delta$ and $\lambda$ (wihtin $\rho$), which we call it ``\textbf{dual separation condition}'' under noise. Note that this reduces to exactly the geometric condition in \cite{soltanolkotabi2011geometric}'s Theorem~2.5 when $\delta=0$.

\subsection{避免平凡解}\label{sec:avoid_trivial}
除了满足 SEP 以外, 我们还要要求问题的解是非平凡的。 不难看出,只要当 $\lambda$
足够大时, 平凡解 $C^* = 0$, $E^*=X_I^{(\ell)}$ 肯定不是最优解。

\begin{lemma}\label{lemma:avoid_trivial}
  对于 \eqref{eq:Opt_A_general} ,当 $\lambda > \frac{1}{\max_{i,j} |X_i^T
Y_j|}$ 时,其最优解一定是非平凡的。
\end{lemma}

\begin{proof}
  我们只要说明存在一个非平凡解比平凡解更优即可。 设 $i, j$ 为 $|X_i^T
  Y_j|$ 取到最大时的下标, 取 $C_{i,j} = a$ 其余取为$0$, 则$C$
  非平凡。我们有
  $$ \frac{\lambda}{2} \|Y_j - X_i a \|_2^2 + |a| = \frac{\lambda}{2}
  \|Y_j\|_2^2 + \frac{\lambda}{2} \|X_i\|_2^2 a^2 +\left( 1-\lambda |X_i^T Y_j|
  \right)a, $$
  其中取 $a = \frac{\lambda |X_i^T Y_j|-1}{\lambda \|X_i\|_2^2} \sign(X_i^T Y_j)$
  则上式成立且等于
  $$ \frac{\lambda}{2} \|Y_j\|_2^2 - \frac{(\lambda|X_i^T Y_j| -1)^2}
  {2 \lambda \|X_i\|_2^2 } < \frac{\lambda}{2} \|Y_j\|_2^2. $$
  因此最优解一定非平凡。
\end{proof}

在对偶分离的条件下,我们只需要考虑在同一个子空间下的点。因此 $\|X_{I^c}^T X_I\|_{\infty} = 
\left\|[X_{I^c}^{(\ell)}]^T X_I\right\|_{\infty}$. 令 $i_0, j_0$ 为矩阵
$[X_{I^c}^{(\ell)}]^T X_I$ 取到最大绝对值的位置, $i,j$ 为矩阵
$[Y_{I^c}^{(\ell)}]^T Y_i$ 取到最大绝对值的位置(如果有不止一个就任选一个),
我们有
\begin{align}
  \left\|[X_{I^c}^{(\ell)}]^TX_I\right\|_{\infty}&=\left|\langle x_{i_0},
  x_{j_0}\rangle\right|  \geq \left|\langle x_i, x_j\rangle\right|\nonumber\\
  &= \left|\langle y_i, y_j\rangle  + \langle y_i, z_j\rangle + \langle z_i, y_j\rangle + \langle z_i, z_j\rangle\right|\nonumber\\
  &\geq \left|\langle y_i, y_j\rangle\right| - \left|\langle y_i, z_j\rangle + \langle z_i, y_j\rangle + \langle z_i, z_j\rangle\right|\nonumber\\
  & = \left\|[Y_{I^c}^{(\ell)}]^Ty_i\right\|_{\infty} - \left|\langle y_i, z_j\rangle + \langle z_i, y_j\rangle + \langle z_i, z_j\rangle\right|\nonumber\\
  &\geq r(\mathcal{Q}^{(\ell)}_{I^c}) - 2\delta - \delta^2. \label{eq:lowerbounding_max_affinity}
\end{align}
其中 $x_i, x_{i_0}$ 表示 $X_{I^c}^{\ell}$ 矩阵的第 $i$ 和 $i_0$ 列,$x_j,
x_{j_0}$ 表示 $X_I^{\ell}$ 的第 $j$ 和 $j_0$ 列, $y_i$ 是 $Y_{I^c}^{\ell}$
的第 $i$ 列, $y_j$ 是 $Y_I^{\ell}$ 的第 $j$ 列。 最后一个不等式成立是因为 $\mathcal{Q}^{(\ell)}_{I^c}$ 
的内接球半径是集合 $\left\{\left\| [ Y_{I^c}^{(\ell)}]^T w\right\|_\infty: w
\in  \mathcal{S}_{\ell}\right\} $ 的下界。因此只要
\begin{equation}\label{eq:lambda_low}
  \lambda > \frac{1}{ r(\mathcal{Q}^{(\ell)}_{I^c}) - 2\delta - \delta^2},
\end{equation}
\eqref{eq:Opt_fictitious2}的解就是非平凡的。 Also, check that
\begin{equation}\label{eq:deterministic_delta}
\delta<\frac{r(r_\ell - \mu_\ell)}{2+7r_\ell}
\end{equation}
under bound of $\delta$ in the theorem statement,  $r(\mathcal{Q}^{(\ell)}_{I^c}) - 2\delta - \delta^2>0$ for any $i,\ell$.

A side remark is that the Lasso regularization path is formally described in \cite{tibshirani2013lasso} and it is unique whenever the data points are in general position. As a result, we can potentially calculate the entry point of $k$th non-zero coefficient for any $0<k<d$, any $x_i$ and $X_{I^c}$. This would however complicate the results unnecessarily, as Lasso path is not monotone (some coefficient may leave the support set as $\lambda$ increases). We therefore stick to the simpler requirement of $c_i$ being non-trivial.

\subsection{存在适当的$\lambda$}\label{sec:exist_lambda}

对于所有 $\ell=1,...,L$,\eqref{eq:Thm1_all} 和 \eqref{eq:lambda_low}
必须被同时满足。 \eqref{eq:lambda_low} 给出了 $\lambda$ 的下界而 \eqref{eq:Thm1_all}
给出了上界。 重复下上面的记号 $r_{\ell}:=\min_{\{i: x_i\in X^{(\ell)}\}}r(\mathcal{Q}^{(\ell)}_{I^c})$,
$\mu_{\ell}:=\mu(\mathcal{X}_{\ell})$ and $r=\min_{\ell} r_{\ell}$, the condition on $\lambda$ is:
%Denote $r_{\ell}:=\min_{\{i: x_i\in X^{(\ell)}\}}r(\mathcal{Q}^{(\ell)}_{I^c})$, $\mu_{\ell}:=\mu(\mathcal{X}_{\ell})$, the condition on $\lambda$ is:
\begin{align*}
  \max_{\ell}\frac{1}{r_{\ell} - 2\delta-\delta^2}<\lambda < \min_{\ell}\frac{r_{\ell}-\mu_{\ell}-2\delta_1}{\delta(1+\delta)(2+r_{\ell}-\delta_1)}.
\end{align*}
%Note that on the left
With the observation that
\begin{align*}
\max_{\ell}\frac{1}{r_{\ell} - 2\delta-\delta^2}
= \frac{1}{\min r_{\ell} - 2\delta-\delta^2},
\end{align*}
% On the right
% $$\min_{\ell}\left\{\frac{r_{\ell}-\mu_{\ell}-2\delta_1}{\delta(1+\delta)(3+2r_{\ell}-2\delta_1)}\right\} = \frac{\min_{\ell}r_{\ell}-\mu_{\ell}-2\delta_1}{\delta(1+\delta)(3+2\min_{\ell}r_{\ell}-2\delta_1)}.$$
%Denote $r=\min_{\ell} r_{\ell}$,
it suffices to require $\lambda$ to obey for each $\ell$:
\begin{equation}\label{eq:lambda_range}
\frac{1}{r - 2\delta-\delta^2}<
        \lambda<\frac{r_{\ell}-\mu_{\ell}-2\delta_1}{\delta(1+\delta)(2+r_{\ell}-\delta_1)}.
\end{equation}



%To understand this, when $\delta$ and $\mu$ is small then any $\lambda$ values satisfying $\Theta(1/r)<\lambda<\Theta(r/\delta)$ will satisfy separation condition.

%We will now derive the condition on $\delta$ such that \eqref{eq:lambda_range} is not an empty set.


We will now show that under condition \eqref{eq:deterministic_delta}, the range \eqref{eq:lambda_range} is not an empty set. Again, we relax $\delta_1$ to $\delta$ in \eqref{eq:lambda_range} and get
\begin{equation}\label{eq:nonempty_cond}
  \frac{1}{r - 2\delta-\delta^2}< \frac{r_{\ell}-\mu_{\ell}-2\delta}{\delta(1+\delta)(2+r_{\ell}-\delta)}.
\end{equation}
Since all denominators are positive, we obtain the standard form of the inequality
$$ A\delta^3+B\delta^2+C\delta+D<0 $$ with
$$
\begin{cases}
A=-3\leq 0\\
B=-3+2(r_\ell-\mu_\ell) + r_\ell \leq 0\\%-(6r-r_{\ell}+7-\mu_{\ell})\\
C=2+4(r_{\ell}-\mu_{\ell})+r_\ell+2r \leq 2+7r_{\ell}\\%2+r_{\ell}+2r+(r_{\ell}-\mu_{\ell})(2r+3)\\%3r_{\ell}r+6r_{\ell}+2r-3\mu_{\ell} r+3-4\mu_{\ell}\\
D=-r(r_{\ell}-\mu_{\ell})
\end{cases}
$$
Check that \eqref{eq:deterministic_delta} is sufficient for the above $3$rd order inequality to hold. Therefore,
$$\eqref{eq:deterministic_delta}\Rightarrow A\delta^3+B\delta^2+C\delta+D<0 \Leftrightarrow \eqref{eq:nonempty_cond}
\Rightarrow \text{\eqref{eq:lambda_range} is not an empty set.}$$
This completes the proof of Theorem~\ref{thm:thm_general}.

\end{document}
