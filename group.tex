\documentclass[main.tex]{subfiles}
\begin{document}
\chapter{Group LASSO方法的正确性}\label{chp:proof_group}
简单起见,本章不考虑噪音。类似于Multitasking 的情形,我们希望能得到
自表示的充分条件。

\section{确定性模型下的分析}
不考虑噪音时,(\ref{eq:group}) 可以简化为
\begin{align}
  \min_{c_i} \|c_i\|_{\Omega,1} \,\, \mbox{s.t.}\,\, X_{-I}c_i = x_i.
  \label{eq:group_nonoise}
\end{align}

为了得到(\ref{eq:group_nonoise})的正确性条件,我们考虑一般的优化问题
\begin{align}
  \P(y,X): \,\, & \min_c \| c\|_{\Omega,1} \\
                & \mbox{s.t.}\,\, Xc = y
  \label{eq:group_primal}
\end{align}
其对偶问题为 
\begin{align}
  \D(y,X): \,\, &\max_{\lambda} \,\, \langle y, \lambda\rangle \\
                & \mbox{s.t.}\,\, \|X^T\lambda\|_{\Omega,\infty} \leq 1
\end{align}
其中$\|x\|_{\Omega, \infty} = \max_{I \in \Omega} \|x_I\|_2$。

我们有如下引理 
\begin{lemma}
  \label{lem:prim_dual}
  若存在 $c$ 满足 $y = Xc$,其支撑集 $S \subseteq T$,并且,
  存在 $\lambda$ 满足
  \begin{align*}
    [X_{S}^T \lambda]_I = \frac{c_I}{\|c_I\|_2} \, \forall I \in \Omega_S\\
    \|X_{T\cap S^c}^T \lambda\|_{\Omega,\infty} \leq 1 \\
    \|X_{T^c}^T \lambda\|_{\Omega,\infty} < 1
  \end{align*}
  那么$\P(y,X)$的所有最优解$c^*$在指标集$T^c$上都为0。
\end{lemma}
\begin{proof}
  证明类似于引理\ref{lemma:OptimalCondition},只需要去除误差项,
  稍作代换即可。
\end{proof}

\subsection{自表示的几何条件}

不是一般性的,假设数据点都在单位球面上。令 $X^\ell$ 为第$\ell$个子空间中的点组成
矩阵,于是
$$ X^\ell = D^{\ell} A^{\ell} ,$$ 
其中 $A^\ell$ 的每一列都在 单位球面$\S^{d_\ell-1}$ 上,于是每个点$x_i$都有一个
对应的$a_i$ 使得$x_i=D^{\ell}a_i$。

同样类似于Multitasking方法,我们构造二元组满足引理\ref{lem:prim_dual}的要求。
令 $c_{i}^{\ell}$ 是原问题 $\P(a_i^{\ell}, A_{-I}^{\ell})$ 的解,
$\lambda_i^{\ell}$ 为对偶问题 $\D(a_{i}^{\ell}, A_{-I}^{\ell})$.
定义$a_{i}^{\ell}$, $i = 1,2,...,n_\ell$ 的对偶方向向量
$$ v_i^\ell = D^{\ell} \frac{\lambda_i^\ell}{\|\lambda_i^\ell \|_2} $$
我们将$c_i^{\ell}$相应扩充(其余位置补零)得到$c_i$,令 $\nu_i^\ell
= D^{\ell} \lambda_i^\ell$,则$(c_i, \nu_i)$引理 \ref{lem:prim_dual}
的条件,除了
$$ \| X^T_I \nu_i^\ell\|_2 < 1 \, \forall I \in \Omega_{-\ell}.$$
上式等价于
$$\| X^T_I v_i^\ell\|_2 \|\lambda_i^\ell\|_2 < 1$$

我们必须给出$\|\lambda_i^\ell$的上界。注意到$\lambda_i^\ell$在集合
$$P=\{ \lambda_i^\ell: \|A_{-i}^{\ell T}\|_{\Omega,\infty}\}$$
由$P$的中心对称性,可知$P$的外接圆半径是$\|\lambda_i^\ell\|_2$的
上界。定义集合 $P^\circ$ 为 
\begin{align*} 
  P^\circ = \{ z: z = A_{-i}^{\ell} b,\,\, : \|b\|_{\Omega,1} \leq 1\},
\end{align*} 
由\cite{ball1997intro_convex_geometry},$R(P) = \frac{1}{r(P^\circ)}$,于是
\begin{align*} 
  \|\lambda_i^\ell \|_2 \leq R(P) = \frac{1}{r(P^\circ)}
\end{align*} 
因此,自表示的充分条件变为
\begin{align} 
  \left\|X_{I}^T v_i^\ell \right\|_{2} \leq r(P^\circ)
  \label{eq:cond1}
\end{align}

\section{半随机生成模型}

我们考虑所谓的\emph{半随机生成模型},即有$n_p$个参考点,
它们类似于第\ref{chp:greedy}章的假设,独立同分布地采样于子空间
球面上的均匀分布。同时有$n_g$个线性变换$L_1, L_2,\ldots,L_{n_g}$,作用在
参考点$w_i$ 上,得到$L_1 w_i, L_2 w_i, \ldots, L_{n_g} w_i$,总共得到
$n_p \times n_g$个新样本点,这些样本点天然就有组别信息(那些同一个参考点
生成的即为一组),于是可以应用Group LASSO的方法得到系数矩阵,进而转化成
邻接矩阵。

\begin{lemma}[\cite{soltanolkotabi2011geometric}中的引理7.5]
\label{infnormlog}
令 $A\in\mathbb{R}^{d_1\times N_1}$ 的每一列采样于
$\mathbb{R}^{d_1}$单位球面上的均匀分布, $\lambda\in\R^{d_2}$ 
采样于 $\R^{d_2}$ 单位球面上的均匀分布,并且于 $A$ 独立。
$\Sigma\in\R^{d_1\times d_2}$ 是一个确定矩阵。
对任意正常数 $\Delta$,我们有
%
$$
\bigl\|A^T\Sigma\lambda\bigr\|_{\ell_\infty} \le4 \bigl(
\log(N_1+1)+\log \Delta+t \bigr)\frac{\|\Sigma\|_F}{\sqrt{d_1}\sqrt {d_2}},
$$
%
以不小于 $1-\frac{4}{(N_1+1)\Delta^2}e^{-2t}$ 的概率成立.
\end{lemma}
%


\end{document}
