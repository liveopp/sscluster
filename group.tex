\documentclass[main.tex]{subfiles}
\begin{document}
\chapter{Group LASSO正则化的正确性}\label{chp:proof_group}
简单起见,本章不考虑噪音。类似于MultiTasking 的情形,我们希望能得到
自表示的充分条件。

\section{确定性模型下的分析}
我们考虑下面的模型。令 $X^\ell$ 为第$\ell$个子空间中的点组成
矩阵,于是
$$ X^\ell = D_\ell A^{\ell} ,$$ 
其中 $A^\ell$ 的每一列都在 单位球面$\S^{d_\ell-1}$ 上。

在无噪音假设下,\ref{eq:group} 可以简化。我们考虑下面的优化问题
\begin{align}
  \P(y,M): \,\, & \min \| c\|_{\Omega} \\
                & \mbox{s.t.}\,\, Mc = y
  \label{eq:group_primal}
\end{align}
其对偶问题为 
\begin{align}
  \D(y,M): \,\, &\max \,\, \la y, \lambda\ra \\
                & \mbox{s.t.}\,\, \|M^T\lambda\|_{\Omega,\infty} \leq 1
\end{align}
其中$\|x\|_{\Omega, \infty} = \max_{I \in \Omega} \|x_I\|_2$。

我们有如下引理 
\begin{lemma}
  \label{lem:prim_dual}
  若存在 $c$ 满足 $y = Mc$,其支撑集 $S \subseteq T$,进一步地,
  存在 $\lambda$ 满足
  \begin{align*}
    [M_{S}^T \lambda]_I = \frac{c_I}{\|c_I\|_2} \, \forall I \in \Omega_S\\
    \|M_{T\cap S^c}^T \lambda\|_{\Omega,\infty} \leq 1 \\
    \|M_{T^c}^T \lambda\|_{\Omega,\infty} < 1
  \end{align*}
\end{lemma}
\begin{proof}
  证明类似于引理\ref{lemma:OptimalCondition},只需要去除误差项,
  稍作代换即可。
\end{proof}

\subsection{自表示的几何条件}
同样类似于Multitasking方法,我们构造二元组满足引理\ref{lem:prim_dual}的要求。
令 $c_{i}^{\ell}$ 是原问题 $\P(a_i^{\ell}, A_{-i}^{\ell})$ 的解,
$\lambda_i^{\ell}$ 为对偶问题 $\D(a_{i}^{\ell}, A_{-i}^{\ell})$.
定义 \emph{group-dual direction} 为 $a_{i}^{\ell}$, $i = 1,2,...,N_\ell$ via 
$$ v_i^\ell = \D^{\ell} \frac{\lambda_i^\ell}{\|\lambda_i^\ell \|_2} $$
类似于Multitaking 情形,将$c_i^{\ell}$相应扩充(其余位置补零),令 $\nu_i^\ell
= D^{\ell} \lambda_i^\ell$。 于是要满足引理 \ref{lem:prim_dual}
的条件,主要要满足 
$$ \| X^T_I \nu_i^\ell\|_2 < 1 \, \forall I \in \Omega_{-\ell},$$
及要求
$$\| X^T_I v_i^\ell\|_2 \|\lambda_i^\ell\|_2 < 1$$
我们首先给出$\|\lambda_i^\ell$的上界。

注意到$\lambda_i^\ell$在集合
$$P=\{ \lambda_i^\ell: \|A_{-i}^{\ell T}\|_{\Omega,\infity\}$$
由$P$的中心对称性,可知$P$的外接圆半径是$\|\lambda_i^\ell\|_2$的
上界。定义集合 $P^\circ$ 为 
\begin{align*} 
  P^\circ = \{ z: z = A_{-i}}^{\ell} b,\,\, : \|b\|_{\Omega,1} \leq 1\},
\end{align*} 
由\cite{ball1997intro_convex_geometry},$R(P) = \frac{1}{r(P^\circ)}$,于是
\begin{align*} 
  \|\lambda_i^\ell \|_2 \leq R(P) = \frac{1}{r(P^\circ)}
\end{align*} 
因此,自表示的充分条件变为
\begin{align} 
  \left\|X_{I}^T v_i^\ell \right\|_{2} \leq r(P^\circ)
  \label{eq:cond1}
\end{align}

\subsection{}
\end{document}
